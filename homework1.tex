\documentclass[11pt,fleqn]{article}
\usepackage[utf8]{inputenc}
\usepackage[T1]{fontenc}
\usepackage[margin=0.75in]{geometry}
\usepackage[parfill]{parskip}
\usepackage{fancyhdr}
\usepackage{amsmath}
\usepackage{amssymb}
\usepackage{amsthm}
\usepackage{algpseudocode}
\usepackage{enumitem}
\usepackage{lastpage}
\usepackage{datetime}
\usepackage{bookmark}
\usepackage{hyperref}
\usepackage{graphicx}

\algnewcommand\algorithmicforeach{\textbf{for each}}
\algdef{S}[FOR]{ForEach}[1]{\algorithmicforeach\ #1\ \algorithmicdo}

\newcommand{\doubleR}{\mathbb{R}}
\newcommand{\doubleC}{\mathbb{C}}
\newcommand{\doubleZ}{\mathbb{Z}}
\newcommand{\doubleQ}{\mathbb{Q}}
\newcommand{\doubleN}{\mathbb{N}}
\newcommand{\RR}{\mathbb{R}}
\newcommand{\CC}{\mathbb{C}}
\newcommand{\ZZ}{\mathbb{Z}}
\newcommand{\QQ}{\mathbb{Q}}
\newcommand{\NN}{\mathbb{N}}
\newcommand{\bigO}{\ensuremath{\mathcal{O}}}

% induction
\newcommand{\Base}{\textbf{Base Step}: }
\newcommand{\IH}{\textbf{Induction Hypothesis}: }
\newcommand{\IS}{\textbf{Induction Step}: }

% theorem and proofs
\newtheorem*{theorem}{Theorem}
% \newtheorem*{statement}{Statement}
\theoremstyle{definition}
\newtheorem*{definition}{Definition}
\theoremstyle{remark}
\newtheorem*{remark}{Remark}

\fancyhf{} % sets both header and footer to nothing
\renewcommand{\headrulewidth}{0pt}
\fancyfoot[C]{Page \thepage}
\pagenumbering{arabic}

\begin{document}

\begin{titlepage}
\title{ECE345 HW1}
\author{ Homework Group \#85 \\
Yanni Alan Alevras | yanni.alevras@mail.utoronto.ca | 1009330706\\
Evan Banerjee | evan.banerjee@mail.utoronto.ca | 1009682309\\
Gautam | gautam.makheja@mail.utoronto.ca | 1008788974}
\date{
Total Pages: \pageref{LastPage}\\[2ex]
\today}
\maketitle

\setcounter{page}{0}
\thispagestyle{empty}
\end{titlepage}
\pagebreak \newpage

\clearpage
\newpage
\pagestyle{fancy}

\section{Question 1, Graphs and Combinations, Permutations}
\begin{enumerate}
    \item[a.]
    \begin{theorem}
        \normalfont Give a combinatorial argument to prove that 
        \[
        \sum_{k=0}^{n} \binom{n}{k} 2^k = 3^n
        \]
    \end{theorem}
    \begin{proof} Combinatorial Argument\\\\
        Imagine you have a bucket with three balls in it, where each ball
        is either red, green, or blue.\\

        let's say each time you pick one of the balls out of the bucket,
        you make a dash on a piece of paper using
        a marker with the same color as the color of the ball.
        Let's also say that you put the ball back after making each dash. Assume that
        you write the dashes in a straight line so that after repeating this process
        n times, you end up with a well defined sequence of n dashes, where 
        each dash is either red, green, or blue.\\

        since each dash was one of three possible choices, there are 
        ${3}^{n}$ possible sequences of colored dashes 
        that you can draw with this process.\\

        Each colored dash has 2 properties, its color, and its index in
        the sequence.\\
        For example, if n = 5, one possible order could be RGGRB.
        in this case, you could represent this unique order as\\\\
        (G, 3), (R, 1), (G, 2), (R, 4), (B, 5)\\\\ 
        where, in each tuple, the letter is the color and
        the number is the index of that color in the sequence.\\

        Clearly the index of each colored dash is going to be a 
        positive integer between 1 and n\\

        Because each unique sequence of n colored dashes can be represented
        by a unique, unordered list of color-index tuples, any procedure that can
        generate this list of tuples is mathematically identical to 
        the process of drawing the sequences by choosing colored balls.\\

        Now imagine you had a bucket with n numbered tokens in it. Each
        token has number written on it ranging from 1 to n.\\ 
        
        One possible way to create an unordered list of color-index 
        tuples (and as a result, a way to create the original sequence
        of colored dashes) would be to first decide how many red tuples
        you wanted ahead of time, create those tuples by randomly selecting
        indices from the bucket, and then randomly choose between
        green and blue as you made the remaining tuples\\

        Again, assuming n = 5, you could start out by saying that you
        will have exactly 2 red tuples. To do this, you could randomly
        choose 2 tokens from the bucket without replacement.\\

        Whatever numbers you pulled out of the bucket, you would create
        red tuples using those numbers as your indices. For example,
        if the numbers you pulled out of the bucket were 1 and 4,
        then you would have the tuples (R, 1) and (R, 4). 
        This exactly defines the red dashes in the sequence presented
        earlier in this proof.\\

        However, there's no reason why the tokens pulled from the bucket
        had to be 1 and 4. Since you're choosing 2 tokens from a 
        set of 5 tokens without replacement, there are $\binom{5}{2}$
        possible combinations 2 red tuples.\\

        After forming these red tuples, there are now 3 indices left.
        Each of these indices could be paired with either 
        the color green, or the color blue.\\

        You could choose to keep pulling numbers out of the bucket,
        assigning each index token that you pull out 
        to either green or blue based on the toss of a coin\\ 
        
        However, another equally valid approach would be to just 
        pour out the remaining tokens, line them up in ascending order,
        and then flip a coin as you look at each token one by one\\

        Heads means the index makes a tuple with green, tails means it
        makes a tuple with blue.\\
        
        If the remaining indices after making
        the red tuples were 2, 3, and 5, and if the three coin tosses
        had given the sequence H, H, T, then we would end up with the
        tuples\\

        (G, 2), (G, 3), (B, 5)\\

        Which represents the blue and green dashes in the example shown
        earlier in this proof.\\

        Because there are 2 possible outcomes for each of the 
        3 remaining tokens (Heads or Tails), the number of possible 
        combinations of green and blue tuples is ${2}^{3}$.\\

        Since there are $\binom{5}{2}$ combinations of red tuples, 
        and ${2}^{3}$ combinations of green and blue tuples,
        there are $\binom{5}{2} * {2}^{3}$ possible combinations of
        5 index-color tuples where 2 have the color red\\

        This means that there are $\binom{5}{2} * {2}^{3}$ 
        possible sequences of 5 colored dashes
        (from the colors red, green, and blue) 
        where two of those dashes are red\\

        However, if we go back to the tuple generation, there was no
        reason why we had to have exactly two red dashes. We could have
        had no red dashes, or five red dashes, or any number in between\\

        Let's define k as the number of red tuples (where $k \in [0, 5]$).
        for each k, there are $\binom{5}{k}$ possible combinations
        of k red tuples, and since there are 5-k remaining indices, we'll end up
        with ${2}^{{5}-{k}}$ unique combinations or green and blue
        tuples. In other words, there are $\binom{5}{k} * {2}^{5-k}$\\

        Every unique combination of tuples will have some amount of 
        red tuples, with the exact amount being between 0 and 5 
        inclusive.\\

        As a result, if we were to sum up all the possible combinations
        of tuples as the number of red tuples (k) 
        goes from 0 to 5, then we would end up with every
        possible combination of 5 tuples formed from the colors red, green,
        and blue.\\

        This sum would look like:\\

        ($\binom{5}{0} * {2}^{5}) + (\binom{5}{1} * {2}^{4}) + 
        (\binom{5}{2} * {2}^{3}) + (\binom{5}{3} * {2}^{2}) +
        (\binom{5}{4} * {2}^{1}) + (\binom{5}{5} * {2}^{0}$)\\

        Which is the same as:

        \[
        \sum_{k=0}^{5} \binom{5}{k} 2^{5 - k}
        \]

        Due to the equivalency of unique tuple combinations and
        unique RGB color sequences, this sum is also equal to the 
        number of unique sequences of 5 colored dashes.\\

        However, there's no reason why we had to choose 5 as our number
        of tuples. This process works just as well for making
        n unique color-index tuples using red, green, and blue.\\

        If we choose to make n unique color-index tuples, and we choose
        to make k red tuples, then there are $\binom{n}{k}$ unique
        combinations of red tuples, and ${2}^{n-k}$ unique combinations
        of green and blue tuples.\\

        this means there are $\binom{n}{k} * {2}^{5-k}$ unique 
        combinations of n tuples where k tuples are red, and 

        \[
        \sum_{k=0}^{n} \binom{n}{k} 2^{n - k}
        \]

        unique combinations of n color-index tuples as a whole\\

        Again, due to the equivalency between tuple combinations 
        and RGB sequences, this formula also describes the number
        of unique sequences formed from red, green, and blue dashes.\\

        And with a little bit of algebra, it's easy to see that:\\ 
        
        \begin{align*}
            \binom{n}{k} &= \frac{n!}{k!(n-k)!} \\
                         &= \frac{n!}{(n-k)!(k)!} \\
                         &= \frac{n!}{(n-k)!(n - n + k)!} \\
                         &= \frac{n!}{(n-k)!(n-(n-k))!} \\
                         &= \binom{n}{n-k}
        \end{align*}

        Which makes intuitive sense because the number of ways
        to choose k elements to use out of a set of n 
        (without replacement) is the same as the number of ways to 
        choose n - k elements not to use out of a set of n 
        (without replacement)\\ 

        Using this fact, we can rewrite 

        \[
        \sum_{k=0}^{n} \binom{n}{k} 2^{n - k}
        \]

        as 

        \[
        \sum_{k=0}^{n} \binom{n}{n - k} 2^{n - k}
        \]

        We can also easily make the substitution $l = n-k$.
        When k is 0, $l=n$, and when k is n, $l=0$. This lets 
        us rewrite the formula as\\

        \[
        \sum_{l=n}^{0} \binom{n}{l} 2^{l}
        \]

        But it doesn't really matter the order in which we evaluate
        a sum. After all, $a + b + c = c + b + a$, so we can say:

        \[
        \sum_{l=n}^{0} f(l) = \sum_{l=0}^{n} f(l)
        \]

        which allows us to rewrite our expression for the number of 
        unique color-index tuples as: 

        \[
        \sum_{l=0}^{n} \binom{n}{l} 2^{l}
        \]

        which is the same as 

        \[
        \sum_{k=0}^{n} \binom{n}{k} 2^{k}
        \]

        Since this is the number of unique combinations
        of color-index tuples for n tuples, it is also the 
        number of unique sequences of n dashes where
        each dash is either red, green, or blue.\\

        Finally, the number of unique RGB sequences of n dashes 
        is $3^n$ (as previously proven) which means that:

        \[
        \sum_{k=0}^{n} \binom{n}{k} 2^{k} = 3^n
        \]

    \end{proof}
    
    \item[b.]
    Time complexity: $\bigO (n)$
    \begin{proof}
    ($\Leftarrow$) Assume by Contradiction,
    
    ($\Rightarrow$)
    \end{proof}
    
\end{enumerate}
\newpage
\clearpage
\section{Question 2 Example Pseudocode}

\begin{algorithmic}[1]
	\Function{Gale-Shapley}{$E$, $S$}
	\State initialize all employers in $E$ and students in $S$ to unmatched
	\While {an unmatched employer with at least one student on its preference list remains}
	\State choose such an employer $e \in E$
	\State make offer to next student $s \in S$ on $e$'s preference list
	\If {$s$ is unmatched}
	\State Match $e$ with $s$ \Comment{$s$ accepts $e$'s offer}
	\ElsIf {$s$ prefers $e$ to their current employer $e'$}
	\State Unmatch $s$ and $e'$ \Comment{$s$ rejects $e'$}
	\State Match $e$ with $s$ \Comment{$s$ accepts $e$'s offer}
	\EndIf
	\State cross $s$ off $e$'s preference list
	\EndWhile
	\State report the set of matched pairs as the final matching
	\EndFunction
\end{algorithmic}


\newpage
\section{Question 3, Asymptotics}
Sort the following 20 functions from asymptotically smallest to asymptotically largest, indicating ties if there are any. You must provide proofs that \textit{justify your answers} to receive full credit! All logarithms are base 2 unless otherwise stated.

\[
\begin{aligned}
& n^{4.5} - (n - 1)^{4.5}, & \quad n \lg \lg n, & \quad n\sqrt{\frac{n}{2}}, & \quad n^{2n}, & \quad \lg(\lg^*n), \\
& n^{({\lg \lg n})/({\lg n})}, & \quad \lg^{(9001)} n, & \quad \lg^*2^{n}, & \quad \lg(n!), & \quad (\lg n)^{\lg n}, \\
& \sum_{i=2}^n \frac{2}{{i}^2-1}, & \quad e^{2n}, & \quad n^{\lg \lg n}, & \quad n^{1337}, & \quad \lg^*{(n/2)}, \\
& (1 + \frac{1}{787898})^{787898n}, & \quad n!, & \quad \pi, & \quad 2^{\lg^*n}, & \quad (\lg n)^{\lg^* n}
\end{aligned}
\]\\

To solve this problem, I will find a tight bound, or an 
upper and lower bound for each function listed.\\

For each bound I find, I will place the corresponding function
in an "Asymptotic Array" that corresponds to that bound.\\

If I find that the bound for the function being analyzed
is the same as the bound for a function
that I analyzed previously, then I will 
place the function in the same Asymptotic Array 
as the previously analyzed function.\\

Any two functions in the same Asymptotic Array have the same 
asymptotic behavior.\\

All asymptotic arrays will be stored in a larger array for convenince.\\

Once all functions are sorted, I will sort the Arrays from smallest
to largest.

I will be using these properties as provided in textbook section 3.3
(c is a constant, k is a constant greater than 1):\\\\
$c << \lg^*n << ({\lg n})^c << n^{c} << k^{n}$

I would also like to begin by proving that for a > 0, $k^{n} << n^{a*n}$

\begin{proof}
    substitute: $n = k^x$ where x is a positive variable

    this means $x = \log_k n$ \\

    and as n $\to \infty$: 
    
    \noindent \[
    \log_k n \to \infty,  
    \quad 1 - a*\log_k n \to -\infty, 
    \quad n(1 - a*\log_k n) \to -\infty
    \]

    for any positive constant a\\

    the proof is as follows:

    \begin{align*}
        &\lim_{n \to \infty} \frac{k^{n}}{n^{a*n}} = 
        \lim_{x \to \infty} \frac{k^{k^{x}}}{{(k^x)}^{a*k^{x}}} = 
        \lim_{x \to \infty} \frac{k^{k^{x}}}{k^{x{a*k^{x}}}} = 
        \lim_{x \to \infty} \frac{1}{k^{({x{a*k^{x}}} - {k^x})}} = \\
        &\lim_{x \to \infty} \frac{1}{k^{{k^{x}(a*{x} - 1)}}} = 
        \lim_{x \to \infty} \frac{1}{k^{{-k^{x}(1 - a*{x})}}} = 
        \lim_{x \to \infty} {k^{{k^{x}(1 - a*{x})}}} = 
        \lim_{n \to \infty} {k^{{n(1 - a*\log_k n)}}} = \\
        &\lim_{m \to -\infty} {k^{m}} = 0
    \end{align*} 

    so $k^n << n^n$
\end{proof}

The final string of inequalities is therefore:

$c << \lg^*n << ({\lg n})^c << n^{c} << k^{n} << n^n$\\

Let's first examine $n^{4.5} - (n - 1)^{4.5}$\\

This is a specific case of the general function $n^{k} - (n - 1)^{k}$
where k is a nonzero constant.\\

By the mean value theorem, if f(x) is continuous and 
continuously differentiable over an interval, then 
$f(x) - f(x - 1) = 1 * f'(x - \epsilon)$ where $0 < \epsilon < 1$\\

And we know from calculus that:

$f(n) = n^{k} \implies f'(x) = kn^{k-1}$

Therefore:\\ 

$n^{k} - (n - 1)^{k} = k(n - \epsilon)^{k-1}$\\

We can also show that if a is any constant: 

\begin{align*}
\lim_{n \to \infty} \frac{n^{k}}{(n-a)^{k}} = 
\lim_{n \to \infty} (\frac{n}{(n-a)})^{k} = 
(\lim_{n \to \infty} \frac{n}{(n-a)})^{k} = 
(\lim_{n \to \infty} \frac{\frac{d}{dn}n}{\frac{d}{dn}(n-a)})^{k} = 
(1)^{k} = 
1
\end{align*}

which means that if f(n) is a polynomial function, f(n-a) and f(n)
are asymptotically equivalent.\\

putting this all together ($0 < \epsilon < 1$):

\begin{align*}
n^{k} - (n - 1)^{k} = 
k(n-\epsilon)^{k - 1} =
4.5\Theta(n^{k - 1}) = 
\Theta(n^{k - 1})
\end{align*}

Our example is just the case when k = 4.5 so

$n^{4.5} - (n - 1)^{4.5} = \Theta(n^{3.5})$

so we have our first category:

\{$\Theta(n^{3.5}): [n^{4.5} - (n - 1)^{4.5}]$\}\\

Now let's look at $n \lg \lg n$

This function is equal to it's asymptotic ($\theta$) form, 
which I will henceforth refer to as being in its "most simplified form".
Since we can't make it simpler, we just make a new category:


\[
\left\{
\Theta(n^{3.5}): [n^{4.5} - (n - 1)^{4.5}], 
\quad \Theta(n \lg \lg n): [n \lg \lg n]
\right\}
\]\\

Now let's look at $n\sqrt{\frac{n}{2}}$

\begin{align*}
    n\sqrt{\frac{n}{2}} =
    \frac{1}{\sqrt{2}}n*n^{0.5} = 
    \frac{1}{\sqrt{2}}n^{1.5} = 
    \Theta({n^{1.5}})
\end{align*}


\[
\left\{
\Theta(n^{3.5}): [n^{4.5} - (n - 1)^{4.5}], 
\quad \Theta(n \lg \lg n): [n \lg \lg n],
\quad \Theta(n^{1.5}): [n\sqrt{\frac{n}{2}}],
\right\}
\]\\

Now let's look at $n^{2n}$

This is already in its most simplified form\\

\[
\left\{
\begin{aligned}
& \Theta(n^{3.5}): [n^{4.5} - (n - 1)^{4.5}], 
& \quad \Theta(n \lg \lg n): [n \lg \lg n],
& \quad \Theta(n^{1.5}): [n\sqrt{\frac{n}{2}}],\\
& \Theta(n^{2n}): [n^{2n}],
\end{aligned}
\right\}
\]\\

Now let's look at $\lg(\lg^*n)$

This is already in its most simplified form\\

\[
\left\{
\begin{aligned}
& \Theta(n^{3.5}): [n^{4.5} - (n - 1)^{4.5}], 
& \quad \Theta(n \lg \lg n): [n \lg \lg n],
& \quad \Theta(n^{1.5}): [n\sqrt{\frac{n}{2}}],\\
& \Theta(n^{2n}): [n^{2n}],
& \quad \Theta(\lg(\lg^*n)): [\lg(\lg^*n)],
\end{aligned}
\right\}
\]\\

The rule I'm using to 
determine whether a function is already in its big theta form is 
this:\\

"If a function is formed entirely of subfunctions from different 
asymptotic bounds, and those functions do not have the capacity
to cancel each other, then the function is in its most simplified 
asymptotic form."

For example, $e^{2n}$ is in its most simplified form because it's 
formed of a polynomial function (2n) 
nested inside an exponential function ($e^n$).
polynomial functions and exponential functions don't have the capacity
to cancel when nested, so this must be fully simplified.\\ 

As another example of a fully simplified function: $nlgn$. It's formed
from the product of a polynomial function and a logarithmic function, and
these functions cannot cancel in a product, so it must already be simplified.\\

However, I cannot assume a function such as $\lg 5^n$ is in its most 
simplified form, because logarithmic functions and exponential functions
have the capacity to cancel, out when nested together, so further 
simplification must occur\\

To save time, I'm going to add all the functions that are already 
equivalent to their big theta forms:

\[
\left\{
\begin{aligned}
& \Theta(n^{3.5}): [n^{4.5} - (n - 1)^{4.5}], 
& \quad \Theta(n \lg \lg n): [n \lg \lg n],
& \quad \Theta(n^{1.5}): [n\sqrt{\frac{n}{2}}],\\
& \Theta(n^{2n}): [n^{2n}],
& \quad \Theta(\lg(\lg^*n)): [\lg(\lg^*n)],
& \quad \Theta(\lg^{(9001)} n): [\lg^{(9001)} n)],\\
& \Theta(1): [\pi],
& \quad \Theta(e^{2n}): [e^{2n}],\\
& \Theta(n^{1337}): [n^{1337}],
& \quad \Theta(2^{\lg^*{n}}): [2^{\lg^*{n}}],
& \quad \Theta({(\lg n)}^{\lg^*{n}}): [{(\lg n)}^{\lg^*{n}}],
\end{aligned}
\right\}
\]\\

before further analysis, let's order this array from smallest
to largest asymptotically when reading from left to right, and top 
to bottom\\

since the smallest asymptotic growth is just a constant value,
$\Theta(1)$ comes first:

\[
\left\{
\begin{aligned}
& \Theta(1): [\pi]
\end{aligned}
\right\}
\]

then, the next growth rate in terms of asymptotic size after $\Theta(1)$
is $\Theta(\lg^*n)$ (we'll verify this later). We have two relevant Theta functions here: 

$\Theta(\lg(\lg^*n))$\\
$\Theta(2^{\lg^*n})$\\

Since $\lg^*n$ just acts as the input into another function, 
so as long as $\lg^*n$ increases without bound:\\

$\lg(\lg^*n) << \lg^*n$ and $2^{\lg^*n} >> \lg^*n$

as a result, the ordering will look like this:\\

\[
\left\{
\begin{aligned}
& \Theta(1): [\pi],
& \quad \Theta(\lg(\lg^*n)): [\lg(\lg^*n)],
& \quad \Theta(2^{\lg^*n}): [2^{\lg^*n})],
\end{aligned}
\right\}
\]\\

The next smallest function asymptotically is going to be $\lg^{(9001)} n$
this is because every other simplified function is at least $\Omega(\lg n)$,
since $\lg{f(n)} << f(n)$ when f(n) increases without bound, 
$\lg^{(9001)} n << lg^{(9000)} n << lg^{(8999)} n << ... << \lg (\lg n) << \lg n$ 

now you may be wondering how we know that $\lg^{(9001)} n >> 2^{\lg^*n}$.

First let's prove that ${}^n 2 >> 2^n$

\begin{proof}
Induction:\\

Base Case: $n_0 = 3$:\\
$2^n = 2^3 = 8$\\
${}^n 2 = 2^{2^2} = 2^4 = 16$\\
16 > 8\\

Inductive Case: ${}^n 2 > 2^n$\\
${}^{(n+1)} 2 = 2^{({}^n 2)}$\\
$2^{(n+1)} = 2 * 2^n$\\

for large values of n, $2^n >> 2n$,

and we know ${}^n 2 > 2^n$, so\\

$\lim_{n \to \infty} \frac{2^{({}^n 2)}}{2*2^n}$ >
$\lim_{n \to \infty} \frac{2^{2^n}}{2*2^n}$ = 0

so ${}^n 2 >> 2^n$
\end{proof}
Now we can prove $\lg^{(9001)} n >> 2^{\lg^*n}$. by setting $n = {}^k 2 = 2^{2^{2^{\cdot^{\cdot^{\cdot 2}}}}}$ with k
2s in total on the tower of 2.

since $\lg2^x = x$, the lg function removes one of the 2s from
the tower of 2s in the tetration. This means that $\lg{({}^k 2)} = {}^{(k-1)} 2$

Then $\lg^*n = \lg^*{({}^k 2)} = k$ (this is basically the definition of $\lg^*$)

if we set $k_0 = 9001$, then $\lg^{(9001)} n$ = $\lg^{(9001)} ({}^k 2) = {}^{(k-9001)} 2$

and shifting a function doesn't change it's asymptotic growth rate relative
to lower order functions (since no matter how large a headstart a lower order
function has, eventually the function with the higher growth rate will dominate), so

${}^{(k - 9001)} 2 >> k =lg^*n$\\\\

\[
\left\{
\begin{aligned}
& \Theta(1): [\pi],
& \quad \Theta(\lg(\lg^*n)): [\lg(\lg^*n)],
& \quad \Theta(2^{\lg^*n}): [2^{\lg^*n})],\\
& \Theta(\lg^{(9001)} n): [\lg^{(9001)} n)],
\end{aligned}
\right\}
\]\\


The next smallest function asymptotically is going to be
${(\lg n)}^{\lg^*{n}}$ (which, just based off inspection, has to be $\Omega(\lg n)$ for large n)

This is because we can prove that ${(\lg n)}^{\lg^*{n}} << n$, and
every other function in the simplified list as of now is at $\Omega(n)$

\begin{proof} Define k such that $n = {}^k 2$
    
    $\lim_{n \to \infty} \frac{{(\lg n)}^{\lg^*{n}}}{n}$ =
    $\lim_{k \to \infty} \frac{{(\lg{({}^k 2)})}^{\lg^*{({}^k 2})}}{{}^k 2}$ =
    $\lim_{k \to \infty} \frac{{({({}^{(k-1)} 2)})}^{k}}{{}^k 2}$\\

    by definition, ${}^k 2 = 2^{({}^{(k-1)} 2)}$ so we can rewrite the 
    previous equation as:

    $\lim_{k \to \infty} \frac{{({({}^{(k-1)} 2)})}^{k}}{2^{({}^{(k-1)} 2)}}$\\

    let's make another substitution $X = \lg n = {}^{(k-1)} 2$ while $n = 2^X$

    the function is now:

    $\lim_{x \to \infty} \frac{{X}^{k}}{2^{X}}$\\

    we know $k = \lg^*n = \lg^*{2^X} = 1 + \lg^*X << \lg^*X$

    which means $\lim_{x \to \infty} \frac{{X}^{k}}{2^{X}} << \lim_{x \to \infty} \frac{{X}^{\lg{X}}}{2^{X}}$\\

    however, in the general problem $\lim_{n \to \infty} \frac{n^{\lg{n}}}{2^n}$
    
    you can substitute $n = 2^t$ to get

    $\lim_{t \to \infty} \frac{{2^t}^{(\lg{2^t})}}{2^{2^t}}$ =
    
    $\lim_{t \to \infty} \frac{{2^t}^{(t)}}{2^{2^t}}$ =

    $\lim_{t \to \infty} \frac{{2^{t^2}}}{2^{2^t}}$ 

    and since $t^2 << 2^t$ -> ${2^{t^2}} << 2^{2^t}$

    since $2^n$ grows without bound. This means that 
    
    $\lim_{t \to \infty} \frac{{2^{t^2}}}{2^{2^t}}$ = 0

    which means that $\lim_{x \to \infty} \frac{{X}^{k}}{2^{X}}$ = 0

    which means that $\lim_{n \to \infty} \frac{{(\lg n)}^{\lg^*{n}}}{n}$ = 0

    which means that ${(\lg n)}^{\lg^*{n}} << n$

    \end{proof}

So now we have:

\[
\left\{
\begin{aligned}
& \Theta(1): [\pi],
& \quad \Theta(\lg(\lg^*n)): [\lg(\lg^*n)],
& \quad \Theta(2^{\lg^*n}): [2^{\lg^*n})],\\
& \Theta(\lg^{(9001)} n): [\lg^{(9001)} n)],
& \mathcal{O}(n)+\Omega(\lg n): [({(\lg n)}^{\lg^*{n}})],
\end{aligned}
\right\}
\]\\

the next function is going to be $n\lg \lg n$ this is because (for a > 0)

$n\lg \lg n << n\lg n << n * n^a = n^b$ where b > 1, and the next smallest function in the 

list has size $n^{1.5}$

\[
\left\{
\begin{aligned}
& \Theta(1): [\pi],
& \quad \Theta(\lg(\lg^*n)): [\lg(\lg^*n)],
& \quad \Theta(2^{\lg^*n}): [2^{\lg^*n})],\\
& \Theta(\lg^{(9001)} n): [\lg^{(9001)} n)],
& \mathcal{O}(n)+\Omega(\lg n): [({(\lg n)}^{\lg^*{n}})],
& \quad \Theta(n\lg \lg n): [n\lg \lg n],
\end{aligned}
\right\}
\]\\

the next three terms are all polynomial, so I will omit proofs since
$n^a >> n^b$ iff $a > b$

\[
\left\{
\begin{aligned}
& \Theta(1): [\pi],
& \quad \Theta(\lg(\lg^*n)): [\lg(\lg^*n)],
& \quad \Theta(2^{\lg^*n}): [2^{\lg^*n})],\\
& \Theta(\lg^{(9001)} n): [\lg^{(9001)} n)],
& \quad \mathcal{O}(n)+\Omega(\lg n): [({(\lg n)}^{\lg^*{n}})],
& \quad \Theta(n\lg \lg n): [n\lg \lg n],\\
& \Theta(n^{1.5}): [n\sqrt{\frac{n}{2}}],
& \quad \Theta(n^{3.5}): [n^{4.5} - (n - 1)^{4.5}], 
& \quad\Theta(n^{1337}): [n^{1337}],
\end{aligned}
\right\}
\]\\

the next functions will be the exponentials, which for now
is only $e^{2n}$. This makes sense because exponentials grow faster
than all polynomials asymptotically:

\[
\left\{
\begin{aligned}
& \Theta(1): [\pi],
& \quad \Theta(\lg(\lg^*n)): [\lg(\lg^*n)],
& \quad \Theta(2^{\lg^*n}): [2^{\lg^*n})],\\
& \Theta(\lg^{(9001)} n): [\lg^{(9001)} n)],
& \quad \mathcal{O}(n)+\Omega(\lg n): [({(\lg n)}^{\lg^*{n}})],
& \quad \Theta(n\lg \lg n): [n\lg \lg n],\\
& \Theta(n^{1.5}): [n\sqrt{\frac{n}{2}}],
& \quad \Theta(n^{3.5}): [n^{4.5} - (n - 1)^{4.5}], 
& \quad \Theta(n^{1337}): [n^{1337}],\\
& \Theta(e^{2n}): [e^{2n}],
\end{aligned}
\right\}
\]\\

the next terms are gonna be the ones that take the form of $n^{an}$

since we already proved those are greater than $k^n$. For now this only 

includes $n^{2n}$

\[
\left\{
\begin{aligned}
& \Theta(1): [\pi],
& \quad \Theta(\lg(\lg^*n)): [\lg(\lg^*n)],
& \quad \Theta(2^{\lg^*n}): [2^{\lg^*n})],\\
& \Theta(\lg^{(9001)} n): [\lg^{(9001)} n)],
& \quad \mathcal{O}(n)+\Omega(\lg n): [({(\lg n)}^{\lg^*{n}})],
& \quad \Theta(n\lg \lg n): [n\lg \lg n],\\
& \Theta(n^{1.5}): [n\sqrt{\frac{n}{2}}],
& \quad \Theta(n^{3.5}): [n^{4.5} - (n - 1)^{4.5}], 
& \quad \Theta(n^{1337}): [n^{1337}],\\
& \Theta(e^{2n}): [e^{2n}],
& \Theta(n^{2n}): [n^{2n}],
\end{aligned}
\right\}
\]\\

let's now decompose n!

$n! = \prod_{k=1}^{n} k$

since $k\leq n$, we can say

$n! = \prod_{k=1}^{n} k \leq  \prod_{k=1}^{n} n = n^n$

$n! = \mathcal{O}(n^n)$

as for the lower bound, we can manipulate the definition of n!:

$n! = \prod_{k=1}^{n} k = n! = 1 \cdot 2 \cdot 3 \cdot \dots \cdot n$ = 

$(1 \cdot n) \cdot (2 \cdot (n-1))  \dots (\frac{n}{2} \cdot (\frac{n}{2} + 1))$

where, for convenience, I have assumed n to be even.
notice, the sum of the two terms in each pair is always n + 1,
and there are $\frac{n}{2}$ terms.

this means we can rewrite n! as 

$\prod_{k=1}^{n/2} (k \cdot ((n + 1) - k))$

if we look at the function $k \cdot ((n + 1) - k)$ (where n is assumed to be 
constant), then we can distribute the k to get 

$(k(n + 1) - k^2)$

this is clearly an upside down parabola, so it's max must occur
at the location where the derivative with respect to k is 0.
This occurs when:

$\frac{d}{dk}(k(n + 1) - k^2) = 0$\\
$((n + 1) - 2k) = 0$\\
$((n + 1) = 2k$\\
$k = \frac{(n + 1)}{2}$\\
$k = \frac{n}{2} + 1/2$

k is never greater than $\frac{n}{2}$, which means that
the derivative $((n + 1) - 2k) \geq ((n + 1) - 2\frac{n}{2}) = 1 > 0$ 

in other words, as k becomes less than $\frac{n}{2}$, the value
if the term $k \cdot ((n + 1) - k$ is always decreasing, so the smallest
possible value it could take would be when k = 1\\

this means the smallest term in the product is n, so:

$n! = \prod_{k=1}^{n/2} (k \cdot ((n + 1) - k))$ >
$\prod_{k=1}^{n/2} n)$ = $n^{n/2}$

so 

$n! = \Omega(n^{n/2})$

since $e^{2n} << n^{n/2}$ and $n^n << n^{2n}$\\

we can say $e^{2n} << n! << n^{2n}$

\[
\left\{
\begin{aligned}
& \Theta(1): [\pi],
& \quad \Theta(\lg(\lg^*n)): [\lg(\lg^*n)],
& \quad \Theta(2^{\lg^*n}): [2^{\lg^*n})],\\
& \Theta(\lg^{(9001)} n): [\lg^{(9001)} n)],
& \quad \Omega(\lg n) + \mathcal{O}(n): [({(\lg n)}^{\lg^*{n}})],
& \quad \Theta(n\lg \lg n): [n\lg \lg n],\\
& \Theta(n^{1.5}): [n\sqrt{\frac{n}{2}}],
& \quad \Theta(n^{3.5}): [n^{4.5} - (n - 1)^{4.5}], 
& \quad \Theta(n^{1337}): [n^{1337}],\\
& \Theta(e^{2n}): [e^{2n}],
& \quad \Omega(n^{n/2}) + \mathcal{O}(n^n): [n!],
& \quad \Theta(n^{2n}): [n^{2n}],
\end{aligned}
\right\}
\]\\

now let's look at $n^{({\lg \lg n})/({\lg n})}$

substitue n = $2^k$ or $k = \lg n$:

\begin{align*}
n^{({\lg \lg n})/({\lg n})} =
{2^k}^{{\lg \lg {2^k}} / \lg {2^k}}=
{2^k}^{{\lg k} / k} = 
2^{{k/k} \cdot \lg k} = 
2^{\lg k} =
k =
\lg n
\end{align*}

so $n^{({\lg \lg n})/({\lg n})} = \Theta(\lg n)$

that puts this function between $\lg^{(9001)} n$ and ${(\lg n)}^{\lg^*{n}}$

also reformatting the glossary of functions is annoying so I'm just gonna turn
it into a straight list


\[
\left\{
\begin{aligned}
& \Theta(1): [\pi],\\
& \Theta(\lg(\lg^*n)): [\lg(\lg^*n)],\\
& \Theta(2^{\lg^*n}): [2^{\lg^*n})],\\
& \Theta(\lg^{(9001)} n): [\lg^{(9001)} n)],\\
& \Theta(\lg n): [n^{({\lg \lg n})/({\lg n})}],\\
& \Omega(\lg n) + \mathcal{O}(n): [({(\lg n)}^{\lg^*{n}})],\\
& \Theta(n\lg \lg n): [n\lg \lg n],\\
& \Theta(n^{1.5}): [n\sqrt{\frac{n}{2}}],\\
& \Theta(n^{3.5}): [n^{4.5} - (n - 1)^{4.5}],\\ 
& \Theta(n^{1337}): [n^{1337}],\\
& \Theta(e^{2n}): [e^{2n}],\\
& \Omega(n^{n/2}) + \mathcal{O}(n^n): [n!],\\
& \Theta(n^{2n}): [n^{2n}],
\end{aligned}
\right\}
\]\\

now let's look at $\lg^*{n/2}$

for large n, $\lg n < n/2 < 2^n$

this means

$\lg^* \lg n < \lg^* n/2 < \lg^* 2^n$

if $n = {}^k 2$

888888888888888888888888888888888888888
\[
\left\{
\begin{aligned}
& \Theta(n^{3.5}): [n^{4.5} - (n - 1)^{4.5}], 
& \quad \Theta(n \lg \lg n): [n \lg \lg n],
& \quad \Theta(n^{1.5}): [n\sqrt{\frac{n}{2}}],\\
& \Theta(n^{2n}): [n^{2n}],
& \quad \Theta(\lg(\lg^*n)): [\lg(\lg^*n)],
& \quad \Theta(\lg^{(9001)} n): [\lg^{(9001)} n)],\\
& \Theta(1): [\pi],
& \quad \Theta(e^{2n}): [e^{2n}],\\
& \Theta(n^{1337}): [n^{1337}],
& \quad \Theta(2^{\lg^*{n}}): [2^{\lg^*{n}}],
& \quad \Theta({(\lg n)}^{\lg^*{n}}): [{(\lg n)}^{\lg^*{n}}],
\end{aligned}
\right\}
\]\\


\[
\begin{aligned}
& n^{4.5} - (n - 1)^{4.5}, & \quad n \lg \lg n, & \quad n\sqrt{\frac{n}{2}}, & \quad n^{2n}, & \quad \lg(\lg^*n), \\
& n^{({\lg \lg n})/({\lg n})}, & \quad \lg^{(9001)} n, & \quad \lg^*2^{n}, & \quad \lg(n!), & \quad (\lg n)^{\lg n}, \\
& \sum_{i=2}^n \frac{2}{{i}^2-1}, & \quad e^{2n}, & \quad n^{\lg \lg n}, & \quad n^{1337}, & \quad \lg^*{(n/2)}, \\
& (1 + \frac{1}{787898})^{787898n}, & \quad n!, & \quad \pi, & \quad 2^{\lg^*n}, & \quad (\lg n)^{\lg^* n}
\end{aligned}
\]\\

$\Theta(\lg^*2^n)$\\
let's assume $n = {}^k 2 = 2^{2^{2^{\cdot^{\cdot^{\cdot 2}}}}}$ with k
2s in total on the tower of 2

then $\lg^*n = k$ (this is basically the definition of $\lg^*$)

As a result, if $\lg^*n$ = k, then 

$\lg^*2^n = \lg^*{2^{{}^k 2}} = \lg^*{2^{2^{2^{2^{\cdot^{\cdot^{\cdot 2}}}}}}}$ with (k + 1) 2s

which evaluates to just k+1, or $\lg^*n + 1$.

since for any value of n, $\lg^*{2^n}$ is only $\lg^*n + 1$, we can say
that $\lg^*{2^n} = \Theta(\lg^*{n})$


\newpage
\clearpage

\section{Question 4 Induction}
\begin{enumerate}
    \item[a.]
    \begin{theorem}
        Some theorem here.
    \end{theorem}
    \begin{proof} Proof by induction
    
    \Base {When $n=1$, is true.}
    
    \IH {Suppose is true}
    
    \IS Consider when 
    \end{proof}
    
    \item[b.]
    Time complexity: $\bigO (n)$
    \begin{proof}
    ($\Leftarrow$) Assume by Contradiction,
    
    ($\Rightarrow$)
    \end{proof}
    
\end{enumerate}
\newpage
\clearpage

\section{Question 5 Probability}
\begin{enumerate}
    \item[a.]
    \begin{theorem}
        Some theorem here.
    \end{theorem}
    \begin{proof} Proof by induction
    
    \Base {When $n=1$, is true.}
    
    \IH {Suppose is true}
    
    \IS Consider when 
    \end{proof}
    
    \item[b.]
    Time complexity: $\bigO (n)$
    \begin{proof}
    ($\Leftarrow$) Assume by Contradiction,
    
    ($\Rightarrow$)
    \end{proof}
    
\end{enumerate}
\newpage
\clearpage

\section{Question 6 Graphs, Proof by Contradiction}
 
    \begin{proof}
        Assume by Contradiction,
        
        Assume \( d(u,w) + d(w,v) < d(u,v) \).
        
        This implies that there is a path from \( u \) to \( v \) via \( w \) whose total distance is less than the direct distance between \( u \) and \( v \).

        However, by the definition of \( d(u,v) \), it is the shortest distance between \( u \) and \( v \). Therefore, no other path, including one through \( w \), can have a smaller distance.

        This contradiction arises from our assumption, so the assumption must be false. Hence, we conclude that \( d(u,w) + d(w,v) \geq d(u,v) \).
    \end{proof}
        
    

\newpage
\clearpage

\section{Question 7 Trees, Proof by induction}
    \textbf{Definition:} A \emph{stable parent} is defined as a node with two leaves. Let:
    \begin{itemize}
        \item \( n \) be the number of nodes,
        \item \( l \) be the number of leaves,
        \item \( p \) be the number of stable parents.
    \end{itemize}

    \noindent \textbf{Proof.} Proof by induction.

    Assume the inductive hypothesis:
    \[
    l_n = p_n + 1 \quad \Rightarrow \quad l_{n+1} = p_{n+1} + 1
    \]

    \noindent We will now consider two cases:

    \textbf{Case 1:} Add a child to a leaf.
    \begin{itemize}
        \item Since the leaf gains a child, while a new leaf is added, we have \( l_{n+1} = l_n \).
        \item Since no node with 1 child gains another child, \( p_{n+1} = p_n \).
        \item Therefore, \( l_{n+1} = l_n = p_n + 1 = p_{n+1} + 1 \).
    \end{itemize}

    \textbf{Case 2:} Add a child to a node with 1 child already.
    \begin{itemize}
        \item Since no leaf stops being a leaf, you have simply added a leaf, so \( l_{n+1} = l_n + 1 \).
        \item Since the node with 1 child gains another child, a new stable parent is formed, so \( p_{n+1} = p_n + 1 \).
        \item Therefore, \( l_{n+1} = l_n + 1 = p_{n+1} + 1 = p_n + 1 + 1 \).
    \end{itemize}

    \noindent \textbf{Base case:} For \( n = 1 \), we have:
    \begin{itemize}
        \item 1 node implies 1 leaf and 0 stable parents.
        \item Thus, \( l_1 = 1 \) and \( p_1 = 0 \).
        \item Therefore, \( 1 = 0 + 1 \), which satisfies \( l_1 = p_1 + 1 \).
    \end{itemize}

    \noindent Q.E.D.
\newpage
\clearpage

\end{document}

