\documentclass[11pt,fleqn]{article}
\usepackage[utf8]{inputenc}
\usepackage[T1]{fontenc}
\usepackage[margin=0.75in]{geometry}
\usepackage[parfill]{parskip}
\usepackage{fancyhdr}
\usepackage{amsmath}
\usepackage{amssymb}
\usepackage{amsthm}
\usepackage{algpseudocode}
\usepackage{enumitem}
\usepackage{lastpage}
\usepackage{datetime}
\usepackage{bookmark}
\usepackage{hyperref}
\algnewcommand\algorithmicforeach{\textbf{for each}}
\algdef{S}[FOR]{ForEach}[1]{\algorithmicforeach\ #1\ \algorithmicdo}

\newcommand{\doubleR}{\mathbb{R}}
\newcommand{\doubleC}{\mathbb{C}}
\newcommand{\doubleZ}{\mathbb{Z}}
\newcommand{\doubleQ}{\mathbb{Q}}
\newcommand{\doubleN}{\mathbb{N}}
\newcommand{\RR}{\mathbb{R}}
\newcommand{\CC}{\mathbb{C}}
\newcommand{\ZZ}{\mathbb{Z}}
\newcommand{\QQ}{\mathbb{Q}}
\newcommand{\NN}{\mathbb{N}}
\newcommand{\bigO}{\ensuremath{\mathcal{O}}}

% induction
\newcommand{\Base}{\textbf{Base Step}: }
\newcommand{\IH}{\textbf{Induction Hypothesis}: }
\newcommand{\IS}{\textbf{Induction Step}: }

% theorem and proofs
\newtheorem*{theorem}{Theorem}
% \newtheorem*{statement}{Statement}
\theoremstyle{definition}
\newtheorem*{definition}{Definition}
\theoremstyle{remark}
\newtheorem*{remark}{Remark}

\fancyhf{} % sets both header and footer to nothing
\renewcommand{\headrulewidth}{0pt}
\fancyfoot[C]{Page \thepage}
\pagenumbering{arabic}

\begin{document}

\begin{titlepage}
\title{ECE345 HW1}
\author{ Homework Group \#85 \\
Yanni Alan Alevras | yanni.alevras@mail.utoronto.ca | 1009330706\\
Evan Banerjee | evan.banerjee@mail.utoronto.ca | 1009682309\\
Gautam | | }
\date{
Total Pages: \pageref{LastPage}\\[2ex]
\today}
\maketitle

\setcounter{page}{0}
\thispagestyle{empty}
\end{titlepage}
\pagebreak \newpage

\clearpage
\newpage
\pagestyle{fancy}

\section{Question 1 Examples of this is a second test}
\begin{enumerate}
    \item[a.]
    \begin{theorem}
        Some theorem here.
    \end{theorem}
    \begin{proof} Proof by induction
    
    \Base {When $n=1$, is true.}
    
    \IH {Suppose is true}
    
    \IS Consider when 
    \end{proof}
    
    \item[b.]
    Time complexity: $\bigO (n)$
    \begin{proof}
    ($\Leftarrow$) Assume by Contradiction,
    
    ($\Rightarrow$)
    \end{proof}
    
\end{enumerate}
\newpage
\clearpage
\section{Question 2 Example Pseudocode}

\begin{algorithmic}[1]
	\Function{Gale-Shapley}{$E$, $S$}
	\State initialize all employers in $E$ and students in $S$ to unmatched
	\While {an unmatched employer with at least one student on its preference list remains}
	\State choose such an employer $e \in E$
	\State make offer to next student $s \in S$ on $e$'s preference list
	\If {$s$ is unmatched}
	\State Match $e$ with $s$ \Comment{$s$ accepts $e$'s offer}
	\ElsIf {$s$ prefers $e$ to their current employer $e'$}
	\State Unmatch $s$ and $e'$ \Comment{$s$ rejects $e'$}
	\State Match $e$ with $s$ \Comment{$s$ accepts $e$'s offer}
	\EndIf
	\State cross $s$ off $e$'s preference list
	\EndWhile
	\State report the set of matched pairs as the final matching
	\EndFunction
\end{algorithmic}


\newpage
\section{Question 3}
\begin{figure}[!htb] 
\caption{Example image \label{fig:eg}}
{This figure shows xxx\bigskip}

% \centering{\includegraphics[width=0.75\linewidth]{example-image}}
\end{figure}
\newpage
\clearpage

\section{Question 4 Induction}
\begin{enumerate}
    \item[a.]
    \begin{theorem}
        Some theorem here.
    \end{theorem}
    \begin{proof} Proof by induction
    
    \Base {When $n=1$, is true.}
    
    \IH {Suppose is true}
    
    \IS Consider when 
    \end{proof}
    
    \item[b.]
    Time complexity: $\bigO (n)$
    \begin{proof}
    ($\Leftarrow$) Assume by Contradiction,
    
    ($\Rightarrow$)
    \end{proof}
    
\end{enumerate}
\newpage
\clearpage

\section{Question 5 Probability}
\begin{enumerate}
    \item[a.]
    \begin{theorem}
        Some theorem here.
    \end{theorem}
    \begin{proof} Proof by induction
    
    \Base {When $n=1$, is true.}
    
    \IH {Suppose is true}
    
    \IS Consider when 
    \end{proof}
    
    \item[b.]
    Time complexity: $\bigO (n)$
    \begin{proof}
    ($\Leftarrow$) Assume by Contradiction,
    
    ($\Rightarrow$)
    \end{proof}
    
\end{enumerate}
\newpage
\clearpage

\section{Question 6 Graphs, Proof by Contradiction}
 
    \begin{proof}
        Assume by Contradiction,
        
        Assume \( d(u,w) + d(w,v) < d(u,v) \).
        
        This implies that there is a path from \( u \) to \( v \) via \( w \) whose total distance is less than the direct distance between \( u \) and \( v \).

        However, by the definition of \( d(u,v) \), it is the shortest distance between \( u \) and \( v \). Therefore, no other path, including one through \( w \), can have a smaller distance.

        This contradiction arises from our assumption, so the assumption must be false. Hence, we conclude that \( d(u,w) + d(w,v) \geq d(u,v) \).
    \end{proof}
        
    

\newpage
\clearpage

\section{Question 7 Trees, Proof by induction}
    \textbf{Definition:} A \emph{stable parent} is defined as a node with two leaves. Let:
    \begin{itemize}
        \item \( n \) be the number of nodes,
        \item \( l \) be the number of leaves,
        \item \( p \) be the number of stable parents.
    \end{itemize}

    \noindent \textbf{Proof.} Proof by induction.

    Assume the inductive hypothesis:
    \[
    l_n = p_n + 1 \quad \Rightarrow \quad l_{n+1} = p_{n+1} + 1
    \]

    \noindent We will now consider two cases:

    \textbf{Case 1:} Add a child to a leaf.
    \begin{itemize}
        \item Since the leaf gains a child, while a new leaf is added, we have \( l_{n+1} = l_n \).
        \item Since no node with 1 child gains another child, \( p_{n+1} = p_n \).
        \item Therefore, \( l_{n+1} = l_n = p_n + 1 = p_{n+1} + 1 \).
    \end{itemize}

    \textbf{Case 2:} Add a child to a node with 1 child already.
    \begin{itemize}
        \item Since no leaf stops being a leaf, you have simply added a leaf, so \( l_{n+1} = l_n + 1 \).
        \item Since the node with 1 child gains another child, a new stable parent is formed, so \( p_{n+1} = p_n + 1 \).
        \item Therefore, \( l_{n+1} = l_n + 1 = p_{n+1} + 1 = p_n + 1 + 1 \).
    \end{itemize}

    \noindent \textbf{Base case:} For \( n = 1 \), we have:
    \begin{itemize}
        \item 1 node implies 1 leaf and 0 stable parents.
        \item Thus, \( l_1 = 1 \) and \( p_1 = 0 \).
        \item Therefore, \( 1 = 0 + 1 \), which satisfies \( l_1 = p_1 + 1 \).
    \end{itemize}

    \noindent Q.E.D.
\newpage
\clearpage

\end{document}

